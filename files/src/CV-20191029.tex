%!TeX program=LuaLaTeX
%!TeX encoding=UTF-8
\documentclass[10pt,a4paper]{moderncv}
\usepackage[margin= 1in]{geometry}\recomputelengths
\usepackage{fontspec}
% \setmainfont{Minion Pro}
% \setsansfont{Myriad Pro}
% \setmainfont{Latin Modern Mono}
% [BoldFont=Inconsolatazi4-Bold]
\setmainfont{CMU Typewriter Text}

\moderncvstyle{classic}
% \moderncvcolor{orange}
\setlength{\hintscolumnwidth}{\widthof{2013.10--2016.12}}

%-------------------------------------------------------------------------------
%            Information
%-------------------------------------------------------------------------------

\firstname{HAO, Yun {\ \LARGE {\fontspec{Noto Sans CJK SC}(郝运)}}}
\familyname{}
\title{\normalsize Last Revised - October 2019}
\address{Arnimallee 3, Room 111}{14195 Berlin, Germany}
\mobile{(+49)~1729067753}
% \phone{}
% \fax{}
\email{haoyun.math@gmail.com}
\homepage{haoyun.github.io/s/}
% \extrainfo{}
% \photo[64pt][0.4pt]{}
% \quote{}

%-------------------------------------------------------------------------------
%            Contents
%-------------------------------------------------------------------------------

\begin{document}

\hypersetup{
pdfauthor={Yun Hao},
pdftitle={},
pdfsubject={},
pdfkeywords={},
pdfproducer={},
pdfcreator={}
}
\maketitle


%-------------------------------------------------------------------------------
%            Profile
%-------------------------------------------------------------------------------

%\section{Profile}
%\cvitem{}{}

%-------------------------------------------------------------------------------
%            Education
%-------------------------------------------------------------------------------

\section{Education}
\cventry{2016.4--2019.8}%
{Dr.\ rer.\ nat}%
{Arithmetic Geometry}%
{Freie Universität, Berlin Mathematical School (BMS)}%
{}%
{
    BMS Phase~II study. Date of defense: May 20, 2019. Grade: \emph{Magna cum
    laude}.\\
    Thesis: \emph{A Simpson correspondence for abelian varieties in positive
    characteristic}.\\
    Supervisors: Hélène Esnault and Michael Gröchenig.
}

\cventry{2013.10--2016.12}
{M.\ Sc.}%
{Mathematics}%
{%
    Humboldt Universität zu Berlin (2013.10--2014.3) \&
    Freie Universität (2015.4--2016.12), BMS
}%
{}%
{%
    BMS Phase~I study. Qualifying exam: Oct.\ 2015.
    Master's degree: Dec.\ 2016. Grade: \emph{1.3 (very good)}\\
    Thesis: \emph{Intersection of a correspondence with a graph of Frobenius}.\\
    Supervisor: Hélène Esnault.
}

\cventry{2009.8--2013.7}%
{B. Sc.}%
{Mathematics}%
{Xi'an Jiaotong University (XJTU)}%
{Xi'an, China}{}% arguments 3 to 6 can be left empty

\section{Publications/Preprints}

\cvitem{1}
{A Simpson correspondence for abelian varieties in characteristic $p>0$.
\href{http://arxiv.org/abs/1910.12405}{arXiv:1910.12405}}

%-------------------------------------------------------------------------------
%            Experience
%-------------------------------------------------------------------------------

\section{Conferences \& Schools}

\cventry{2019.9}{Motives and Stacks}{}
    {Universität Duisburg-Essen}{Essen, Germany}{}
\cventry{2019.7}{Arithmetic of connections}{}
    {Monte Verit\`a}{Ascona Switzerland}{}
\cventry{2019.6}{GAeL XXVII}{}
    {IMAR}{Bucharest, Romania}{poster presentation}
\cventry{2019.2--2019.5}{Derived Algebraic Geometry}{Program Associate}
    {MSRI}{Berkeley, CA, US}
    {\httplink[Seminar talk: A Simpson Correspondence for Abelian
    Varieties in Characteristic $p > 0$.]{https://www.msri.org/seminars/24098}}
\cventry{2018.11}{Chow Lectures: by Peter Scholze}{}
    {MPI MiS}{Leipzig, Germany}{poster presentation}
\cventry{2018.9}{Arithmetic of Differential Equations}{}
    {Ośrodek Kolonijno-Wczasowy Bajka}{Łukęcin, Poland}{}
\cventry{2018.9}{International summer school on Arithmetic geometry}{}
    {Università degli Studi di Salerno}{Salerno, Italy}{}
\cventry{2018.4}{Crystals and Geometry in Characteristic $p$}{}
    {TU München}{Munich, Germany}{}
\cventry{2018.2}{Riemann-Hilbert Correspondences}{}
    {Università degli Studi di Padova}{Padua, Italy}{}
\cventry{2017.10}{Topics in arithmetic and algebraic geometry}{}
    {Johannes Gutenberg-Universität Mainz}{Mainz, Germany}{}
\cventry{2017.8}{Motives for periods}{}
    {Freie Universität Berlin}{Berlin, Germany}{}
\cventry{2017.7}{Higgs Bundles, K3 Surfaces and Moduli}{}
    {Humboldt Universität zu Berlin}{Berlin, Germany}{}
\cventry{2016.9}{Higgs Bundles and Fundamental Goups of Algebraic Varieties}{}
    {Universität Duisburg-Essen}{Essen, Gemany}{}
\cventry{2016.6}{Shimura Varieties}{}
    {Leiden University}{Leiden, Netherlands}{}
\cventry{2015.7}{AMS Summer Institue in Algebraic Geometry}{}
    {University of Utah}{Salt Lake City, UT, USA}{}
%\cventry{2014.3}{Jürgen-Ehlers-Frühjahrsschule ``Gravitationsphysik’’ 2014}{Max Planck Institute for Gravitational Physics}{Potsdam-Golm, Germany}{}{}
\cventry{2013.2--2013.6}{Enhanced Program for Graduate Study}{}
    {Peking University}{Beijing, China}{}
%\cventry{2010.6--2010.8}{Research Experience for Undergraduate Students}{The University of Iowa}{Iowa City, IA, USA}{}{}


%-------------------------------------------------------------------------------
%           Teaching
%-------------------------------------------------------------------------------

\section{Teaching}
\cventry{2015.5--2015.7}{Differential Geometry I}{tutorial sessions}{TU
Berlin}{Berlin, Germany}{}

%-------------------------------------------------------------------------------
%           Honors & Awards
%-------------------------------------------------------------------------------

% \section{Honors \& Awards}
% \cventry{2016--2019}{BMS Phase II Scholarship}{BMS}{Berlin, Germany}{}{}
% \cventry{2013--2015}{BMS Phase I Scholarship}{BMS}{Berlin, Germany}{}{}

%-------------------------------------------------------------------------------
%           Languages
%-------------------------------------------------------------------------------

% \section{Languages}
% \cvitem{}{Chinese (native), English (fluent), German (very basic level) and
% French (\textit{only be able to read math literatures})}
% \cvitemwithcomment{Chinese}{\normalfont Native speaker}{}
% \cvitemwithcomment{English}{\normalfont Fluent}{}  %\\ I've read lots of math book in English}
% \cvitemwithcomment{German}{\normalfont Beginner}{}
% \cvitemwithcomment{French}{\normalfont be able to read mathematical literatures.}{}
%\cvitemwithcomment{French}{Beginner}{Start learning from this year}
%\cvitemwithcomment{German}{\normalfont A little bit}{}

%-------------------------------------------------------------------------------
%           Languages
%-------------------------------------------------------------------------------

\section{References}

\renewcommand*{\cvdoubleitem}[5][.25em]{%
\cvitem[#1]{#2}{%
\begin{minipage}[t]{\doubleitemcolumnwidth}#3\end{minipage}%
%\hfill% fill of \separatorcolumnwidth
\begin{minipage}[t]{\hintscolumnwidth}\raggedleft\hintstyle{#4}\end{minipage}%
%\hspace*{\separatorcolumnwidth}%
\begin{minipage}[t]{\doubleitemcolumnwidth}#5\end{minipage}}}

\cvdoubleitem{}{
    \textbf{Hélène Esnault} \\
    Freie Universität Berlin\\
    \emaillink{esnault@math.fu-berlin.de}
}
{}{
    \textbf{Michael Groechenig}\\
    University of Toronto\\
    \emaillink{michael.groechenig@utoronto.ca}
}


%-------------------------------------------------------------------------------
%           Computer Skills
%-------------------------------------------------------------------------------

%\section{Computer Skills}
%\cvitem{Math. Software}{Good at \textsc{Matlab}, using Maple and Mathematica occasionally, with basic knowledge of Scilab, Octave, Sage, etc.}
%\cvitem{Office}{Proficient in \LaTeX{} and skillful in MS Word, Excel and PowerPoint}
%\cvitem{System}{Using Linux (Fedora) as everyday working platform and MS Windows alternatively}
%\cvitem{Programming}{C, GNU BASH, etc.}
%\cvitem{Misc.}{gnuplot, vim, git, etc.}

%-------------------------------------------------------------------------------
%           Interests
%-------------------------------------------------------------------------------

%\section{Interests}
%\cvitem{Computers}{I am an active user and enthusiastic promoter of \TeX/\LaTeX\ in the Chinese \TeX\ community. I wrote a tutorial \textit{Beginner's guide to XeLaTeX and WinEdt} (in Chinese) and it's widespread and highly appreciated.}
%\cvitem{Reading}{I read extensively, involving science and literature.}

\end{document}

% EOF

Quick Reference:

% makes a cv line with a header and a corresponding text
% usage: \cvitem[spacing]{header}{text}
\newcommand*{\cvitem}[3][.25em]{}

% makes a cv line 2 headers and their corresponding text
% usage: \cvdoubleitem[spacing]{header1}{text1}{header2}{text2}
\newcommand*{\cvdoubleitem}[5][.25em]{}

% makes a cv line with a list item
% usage: \cvlistitem[label]{item}
\newcommand*{\cvlistitem}[2][\listitemsymbol]{}

% makes a cv line with 2 list items
% usage: \cvlistdoubleitem[label]{item1}{item2}
\newcommand*{\cvlistdoubleitem}[3][\listitemsymbol]{}

% makes a typical cv job / education entry
% usage: \cventry[spacing]{years}{degree/job title}{institution/employer}{localization}{optionnal: grade/...}{optional: comment/job description}
\newcommand*{\cventry}[7][.25em]{}

% makes a cv entry with a proficiency comment
% usage: \cvitemwithcomment[spacing]{header}{text}{comment}
\newcommand*{\cvitemwithcomment}[4][.25em]{}

% makes a generic hyperlink
% usage: \link[optional text]{link}
\newcommand*{\link}[2][]{%
  \ifthenelse{\equal{#1}{}}%
    {\href{#2}{#2}}%
    {\href{#2}{#1}}}

% makes a http hyperlink
% usage: \httplink[optional text]{link}
\newcommand*{\httplink}[2][]{%
  \ifthenelse{\equal{#1}{}}%
    {\href{http://#2}{#2}}%
    {\href{http://#2}{#1}}}

% makes an email hyperlink
% usage: \emaillink[optional text]{link}
\newcommand*{\emaillink}[2][]{%
  \ifthenelse{\equal{#1}{}}%
    {\href{mailto:#2}{#2}}%
    {\href{mailto:#2}{#1}}}

